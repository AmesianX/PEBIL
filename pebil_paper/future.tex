\section{Future Work}
\label{sec:Future}

Despite the success in terms of efficiency, there are several additional
techniques that might make PEBIL even more efficient. One such technique would be
to inline the instrumentation code itself in order to reduce or eliminate the
control interruptions  that otherwise must be taken when inserting the
instrumentation code. This could be accomplished because PEBIL relocates the
text to yield extra space for the manipulation of the application functions
leaving space for inlining instrumentation snippets, rather than 
inserting a branch that transfers control to the instrumentation code. 

Another technique would be to reduce the saving and storing of registers.
Currently PEBIL saves all general purpose registers around each function call
and allows the tool developer to state which registers are saved around
instrumentation snippets. For even more efficient instrumentation snippets, we
could automatically detect which registers are killed by the instrumentation
code and which are live at the entry point of the instrumentation code, and
automatically save only the ones that are alive. Similarly, we could perform
register analysis in order to identify the instrumentation points where the
machine state doesn't need to be saved around instrumentation functions. 

Finally, similar to Pin, we could perform liveness on the bits of the
eflags/rflags register to determine whether the flag registers need to be saved
and restored at each instrumentation point. Optimizations that help PEBIL avoid
saving and restoring state at each instrumentation point have the potential to
further the overhead associated with instrumentation and we believe that they
will further the goal of generating efficient instrumented code with PEBIL.


