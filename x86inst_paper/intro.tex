\section{Introduction}

Binary instrumention is the act of inserting extra code into a compiled
application in order to observe or modify something about its behavior. The data
collected from binary instrumentation can be useful for guiding hardware and
system design decisions, program debugging, compiler correctness/optimization,
program debugging and security testing/verification.

There are two common approaches to binary instrumentation: static binary
instrumentation and dynamic binary instrumentation. Static binary
instrumentation toolkits have the advantage of usually being able to produce
more efficient instrumented code than comparable dynamic binary instrumentation toolkits
because static instrumentation introduces only the instrumentation code itsself into the run of the
application. Static binary instrumentation tools have some disadvantages,
however. It is not possible to instrument shared libraries or dyanamically
generated code with static binary instrumentation tools. They also provide less
flexibility to the tool developer, since any instrumentation code that is
inserted into the application will persist throughout the application run.
However, there are cases where efficiency is of enough importance to outweigh
these shortcomings in such a way that static binary instrumentation is the
desirable paradigm.

x86elfinstrumentor is a static binary instrumentation toolkit for Linux on
X86/X86\_64 platforms. The goal of x86elfinstrumentor is to provide the ability
to build instrumentation tools that produce very efficient instrumented
applications. We instrument the binary by placing an unconditional branch at each instrumentation
point that transfers control to instrumentation code. This instrumentation code saves the
program state, performs some tasks that are determined by the particular instrumentation tool,
restores the program state, then returns control to the application.
A typical binary instrumentation tool on a platform with fixed-length instructions 
\cite{tikir2006pmac} accomplishes this initial control transfer by replacing a
single instruction at the instrumentation point with a branch that transfers
control to the instrumentation code. However when instructions are variable-length, as
is the case for X86/X86\_64, this is not always possible since the branch instruction can be larger than the
instruction at the instrumentation point. To address this, x86elfinstrumentor
relocates and reorganizes the code for each function to ensure that enough empty
space (in the form of \begin{it}nops\end{it}) is available to hold a full-length branch instruction at each
instrumentation point.

Instrumented code efficiency is accomplished in several ways. Dynamic binary instrumentation
is costly because the instrumentation tool must frequently disrupt the application in order
to perform tasks such as parsing, disassembly, code generation, and other decisions. This
interruption is simply not an issue with static binary instrumentation tools because all
decisions and actions are taken prior to runtime. The only cost born at runtime is the direct
cost of performing instrumentation-related functions. We relocate the application's
functions, which affords us the opportunity to reorganize the code so that we
can use large yet efficient instructions to transfer control from the
application to the instrumentation code. We also use the concept of an
instrumentation snippet, a lightweight hand-written body of assembly code that can
be inserted into the application rather than relying only on heavyweight
instrumentation functions to accomplish instrumentation tasks.

x86elfinstrumentor is open source and available to the public for download along with several
instrumentation tools at \url{http://blind-review-forbids-real-urls.com/}. These tools include a basic block
execution counter, a function execution counter, and a memory address stream collection tool.

The rest of the paper is organized as follows. Section
\ref{Section:Overview} descries the design and implementation of our static
instrumentation toolkit. Section \ref{Section:Efficiency} discusses in greater detail
several aspects of the toolkit, including the function relocation mechanism and the
instrumentation snippet concept. Section \ref{Section:Tools} shows the details of
the instrumentation tools included in the package. Section \ref{Section:Results}
presents a comparison and discussion of the performance of x86elfinstrumentor relative
to other state of the art instrumentation toolkits. Section
\ref{Section:Future} discusses some ideas for the future of x86elfinstrumentor,
and Section \ref{Section:Conclusions} concludes.
