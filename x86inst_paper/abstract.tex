\begin{it}
Binary instrumentation toolkits for the x86/x86\_64 platforms in use today have a tendancy
to use dynamic instrumentation approaches. While these toolkits
are often well-designed and effective, they suffer from a large deficiency; they
are ineffient because they perform all of their analysis, disassembly, and
instrumentation at runtime.

In this work we present X86ElfInstrumentor, a static binary instrumentation
toolkit for Linux on x86/x86\_64 platforms that relies on the use of wholesale
code relocation in order to remedy the difficulty created by the platforms' use
of variable-length instructions. Code relocation of this kind allows the
instrumentation tool to reorganize the application code in such a way that it
can use the fast but far-reaching ``long'' jmp instruction to transfer control
from the application to the instrumentation code rather than relying on multiple
jumps or interrupts for the transfer. This technique leads to efficient
instrumentation tools, with overheads for basic block counting that are an
average of 48\% of the overhead imposed by Pin, 18\% of the overhead imposed by
DynamoRIO, 10\% of the overhead of Valgrind, and 5\% of the overhead of Dyninst.
\end{it}
