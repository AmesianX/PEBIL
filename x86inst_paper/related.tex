ATOM [CITE] was one of the first and has remained one of
the more popular static binary instrumentation tools
available. ATOM works in a way that is very similar to
PMaCinst; instrumentation is performed on the compiled
binary prior to runtime, meaning that any overhead due to
code analysis and code generation is incurred outside of
the instrumented application’s runtime. Unfortunately,
ATOM is available only for the Alpha platform. Since this
processor is not being produced anymore, ATOM is no
longer viable as a long-term solution for those who wish to
perform static, efficient instrumentation on a RISC-based
platform. 

Pin[CITE] is a dynamic binary instrumentation tool that
uses a JIT-based (Just In Time compilation) approach to
instrumentation. This approach entails running the
application on top of Pin, while Pin intercepts the
application at a natural control flow interruption in the
program to perform instrumentation on the next part of the
program. For efficiency, Pin does many things including
caching these instrumented sequences of code to allow for
re-use and chaining instrumented sequences of code
together to avoid tool intervention.

Dyninst[CITE] is another popular dynamic
instrumentation tool that uses a technique called codepatching
to perform instrumentation. Similar in concept to
what is done in PMaCinst, this technique replaces an
instruction from the application with a jump instruction
(which they call a trampoline) to the function call stub and
instrumentation code. The key difference between
PIX and Dyninst is that Dyninst performs all codepatching
at runtime instead of prior to runtime. This has
several advantages, including the ability to insert, remove
and customize instrumentation during runtime. But
performing modification to the program at runtime also
may have a significant performance disadvantage, resulting in
inefficient execution of the instrumented application.

DynamoRIO

Valgrind
